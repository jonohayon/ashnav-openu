\documentclass[12pt, oneside]{article}
\usepackage{geometry}
\geometry{letterpaper}
\usepackage{amssymb}
\usepackage{amsmath}
\usepackage{hyperref}
\usepackage{cancel}

% Initializing graphicx
\usepackage{graphicx}
\graphicspath{{assets/}}

% Custom Commands
\newcommand{\img}[1] {
	\includegraphics[scale=0.33]{#1}
}

\title{Maman 15}
\author{Jonathan Ohayon}
% No date cuz I'm living on the 3dge

\begin{document}
\maketitle

\begin{center}
\framebox{\vbox{\textbf{WARNING:} I used some Hebrew terms in free translation to English (such as fixed points for "nekudot shevet") because I wasn't able to find their actual translation in the internet.}}
\end{center}

\section{Question 1}

\setcounter{subsection}{-1}
\subsection{Declarations}
Let $f:\mathbb{Q} \rightarrow \mathbb{Q}\backslash\{-1\}$ and $g$ be the following function:
\begin{equation*}
\text{For every}\;x \in \mathbb{Q},\;g(x) = \dfrac{f(x)}{f(x) + 1}
\end{equation*}

\subsection{A}
The $g$ function is defined from $\mathbb{Q}$ to $\mathbb{Q}\backslash\{1\}$ ($g: \mathbb{Q} \rightarrow \mathbb{Q}\backslash\{1\}$) because of the fact that in order to produce the number 1 using a fraction, both the numerator and the denominator \emph{have} to be equal. But, according to the declaration of $g$, the denominator always equals to the numerator plus 1. Therefore, $g: \mathbb{Q} \rightarrow \mathbb{Q}\backslash\{1\}$.

\subsection{B}
In order to show that $g$ is injective, we'll presume that $g(a) = g(b)$ and show that $a = b$.
\begin{eqnarray*}
& \frac{f(a)}{f(a) + 1} = \frac{f(b)}{f(b) + 1}\\
& f(a) \cdot (f(b) + 1) = f(b) \cdot (f(a) + 1)\\
& \cancel{f(a)f(b)} + f(a) = \cancel{f(b)f(a)} + f(b)\\
& f(a) = f(b)\\
& \text{Because $f$ is injective, }\framebox{$a = b$}
\end{eqnarray*}
Therefore, if $f$ is injective, $g$ is injective as well.
\clearpage

\subsection{C}
\setcounter{subsubsection}{-1}
\subsubsection{Declarations}
Let $f$ be a surjective function and $y \in \mathbb{Q}\backslash\{1\}$.

\subsubsection{Objective}
In order to show that $g$ is a surjective function as well, we need to show that there is $x \in \mathbb{Q}$ that exists so that $g(x) = y$.

\subsubsection{Actual Proof}
Since we are looking for $x \in \mathbb{Q}$, we can assume (according to the question's declarations) that $y = \frac{f(x)}{f(x) + 1}$, or in other words that $f(x) = y(f(x) + 1) \Rightarrow y = (y  - 1)f(x) \Rightarrow f(x) = \frac{y}{1 - y}$, which means that $x \in \mathbb{Q}$ only exists if $\frac{y}{1 - y}$ is in $\mathbb{Q}\backslash\{-1\}$. First we'll show that if $y \in \mathbb{Q}\backslash\{1\}$, then $\frac{y}{1 - y} \in \mathbb{Q}\backslash\{-1\}$. Because $y$ is a rational number that is different than 1 and $\frac{y}{1 - y}$ is a rational number too. If we'll assume that $\frac{y}{1 - y} = -1$, we'll see that $y = y - 1$, which is a contradiction to the question's declaration - which means that $\frac{y}{1 - y} \in \mathbb{Q}\backslash\{-1\}$. That means that it's inside of $f$'s range. Because $f$ is a surjective function, $x \in \mathbb{Q}$ exists so that $f(x) = \frac{y}{1 - y}$. If we'll try to separate $y$ from the equation we'll find that $y = (y - 1)f(x) \Rightarrow f(x) = y(1 + f(x))$. Because $f(x) \neq -1$ (according to the declaration), we'll see that $y = \frac{f(x)}{f(x) + 1}$. Therefore, $g(x) = y$.

\subsubsection{Conclusion}
In conclusion, we showed that for every $x \in \mathbb{Q}$ there is a $y \in \mathbb{Q}\backslash\{1\}$ so that $g(x) = y$, which means that $g$ is surjective.

\subsection{D}
If $f$ is inverse it is bijective as well, which means (according to sections B and C of the question) that $g$ is bijective as well - which means that it's inverse as well.\\
We can show that by using the following proof:\\
Let $y \in \mathbb{Q}\backslash\{1\}$. Since $g$ is surjective, there is $x \in \mathbb{Q}$ so that $g(x) = y$, and because it's injective as well, $x$ is the only element of $\mathbb{Q}$ of which $g(x) = y$. Because of the declaration of $g$, $y = \frac{f(x)}{f(x) + 1} \Rightarrow f(x) = \frac{y}{1 - y}$. Because $f$ is inverse, $x = f^{-1}(\frac{y}{1 - y})$. Therefore, $g^{-1}(y) = f^{-1}(\frac{y}{1 - y})$.
\clearpage

\section{Question 2}
\setcounter{subsection}{-1}
\subsection{Declarations}
Let $T$ be a set of the points on a circle named $O$ (excluding the surface of $O$). Let $f$ be an isometry so that $T$ is a constant set over it. Let $A, B \in T; d(A, B) = 2R$ aka $A, B$ are edges of a diameter of $O$.

\subsection{A}
Because $T$ is a constant set over $f$, $f(A), f(B) \in T$. Also, because $f$ is an isometry, $d(A, B) = d(f(A), f(B))$. Therefore, $d(f(A), f(B)) = 2R$ because $d(A, B) = 2R$.

\subsection{B}
$O$ is the middle of $AB$ and the middle of $f(A)f(B)$ is $f(O)$. Because $AB = f(A)f(B)$, $Af(O) = f(O)B = AO = BO$. Therefore, $f(O) = O$.

\subsection{C}
\begin{itemize}
\item $f$ is not a translation nor a translated reflection since it has more than 0 fixed points.
\item $f$ is not a rotation since it has more than 1 fixed points.
\item $f$ is not the id function because of the question's declaration.
\item \framebox{$f$ is a reflection.}
\end{itemize}
Therefore, $f$ is a reflection of which the reflection line is $AB$.

\subsection{D}
Let $E$ be the set of all of the points in the surface of circle $O$. For every $X \in E$, let $Y \in AB$ be a point on $AB$ in front of $X$. Because $f$ is an isometry, $d(X, Y) = d(f(X), f(Y))$. But since $Y \in AB$, $f(Y) = Y$, which means $d(X, Y) = d(f(X), f(Y)) = d(f(X), Y)$. Therefore, $f(X) \in E$ and $E$ is a constant set.
\clearpage

\section{Question 3}
\setcounter{subsection}{-1}
\subsection{Declarations}
Let $f, g$ be isometries of a given surface and $A, B$ different points on that surface. It is known that $A, B$ are fixed points of the isometry $f \circ g$.

\subsection{A}
If $g$ is a reflection and $f$ is the id function (or vise versa), then $f \circ g$ doesn't have to be the id function.

\subsection{B}
Because of the fact that it is stated in the question that both $f$ and $g$ don't keep the "triangle trend"\footnote{My free translation to the phrase "megamat meshulashim" since I wasn't able to find an actual one anywhere.}, they cannot be either the id function, a rotation nor a translation. Therefore, they are either regular reflections or translated reflections.

\begin{itemize}
\item Since there are more than 0 fixed points, both $f$ and $g$ have to be the same thing (if not - $f \circ g$ couldn't be the id function nor a reflection).
\item If both of them are reflections, they are inverted because they are on the same line (they have 2 fixed points).
\item If both of them are translated reflections, the have to be inverted because if they weren't there would have not been any fixed points.
\end{itemize}

\subsection{C}
Because $f$ and $g$ don't keep the "triangle trend"\footnote{Again, same translation.} they can be either a translated reflection or a regular reflection. However, since $f$ has a fixed point, $f$ is a regular reflection. Also, because $f \circ g$ is either the id function or a reflection, $g$ has to be either the id function itself or a reflection. Since the id function is not an option in this case, $g$ has to be a reflection and $f = g$.
\clearpage

\section{Question 4}
\setcounter{subsection}{-1}
\subsection{Declarations}
Let $l_1, l_2, l_3$ be three lines, of which $l_2, l_3$ are parallel to each other and they are both vertical to $l_1$. Also, $f = S_{l_3} \circ S_{l_2} \circ S_{l_1}$.

\subsection{A}
We'll mark the point where $l_2$ intersects with $l_1$ using the letter $O$. Because $l_2$ is vertical to $l_1$, the angle between them equals $90^{\circ}$ and $S_{l_1} \circ S_{l_2}$ is a rotation of $180^{\circ}$ on $O$ in the direction from $l_1$ to $l_2$. In the same way, $S_{l_2} \circ S_{l_1}$ is a rotation of $180^{\circ}$ on $O$ in the direction from $l_2$ to $l_1$. But because they unite, $S_{l_2} \circ S_{l_1} = S_{l_1} \circ S_{l_2}$. We can show that $S_{l_1} \circ S_{l_3} = S_{l_3} \circ S_{l_1}$ in the same way, just by replacing $S_{l_2}$ with $S_{l_3}$.

\end{document}