\documentclass[12pt, oneside]{article}
\usepackage{geometry}
\geometry{letterpaper}
\usepackage{amssymb}
\usepackage{amsmath}

\title{Maman 11}
\author{Jonathan Ohayon}

\begin{document}
\maketitle

\section{Question 1}
\subsection{A}
Yes. Using the injective function:
\begin{equation*}
f(x) \rightarrow \frac{2}{x/2}
\end{equation*}
Every $x \in A$ has one arrow sent from it to $\dfrac{2}{x/2} \in B$ (according to the injective function).

\section{Question 2}

\section{Question 3}
\subsection{A}
Yes. Let \emph{A} and \emph{B} be the following sets:
\begin{equation*}
A = \{1, 2, 3, 4\}, B = \{1, 2\}
\end{equation*}
Then:
\begin{equation*}
\begin{split}
& A \cup B = \{1, 2, 3, 4\}\\
& A \backslash B = \{3, 4\}\\
& A \cup B \neq A \backslash B
\end{split}
\end{equation*}\\
Therefore, \emph{B} has to be $\emptyset$.\\
$\bigotimes$

\subsection{B}
No. Let \emph{A} and \emph{B} be the following sets:
\begin{equation}
A = \mathbb{N}, B = \{2, 4, 6, \ldots\}
\end{equation}
Then $A \cup B \sim A \backslash B$, using the following injective function:
\begin{equation*}
f(x) = 2x
\end{equation*}
Every $x \in A$ has one arrow sent from it to $2x \in B$, and every $y \in B$ has an arrow from $\frac{1}{2}y \in A$.\\
Therefore, \emph{B} doesn't have to be $\emptyset$.

\subsection{C}
Yes. Let \emph{A}, \emph{B}, \emph{C} and \emph{D} be the following sets:
\begin{equation*}
\begin{split}
& A = \{1, 2, 3, 4, 5\}\\
& B = \{3, 4, 5\}\\
& C = A \cup B = \{1, 2, 3, 4, 5\}\\
& D = A \backslash B = \{1, 2\}
\end{split}
\end{equation*}
Then $D \nsim C$ because no injective function could match \emph{D}'s members with \emph{C}'s members, since the number of members in \emph{D} is smaller than the number of members in \emph{C}.\\
Therefore, \emph{B} has to be $\emptyset$.\\
$\bigotimes$

\section{Question 4}
\subsection{A}


\end{document}