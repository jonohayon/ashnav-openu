\documentclass[12pt, oneside]{article}
\usepackage{geometry}
\geometry{letterpaper}
\usepackage{amssymb}
\usepackage{amsmath}
\usepackage{hyperref}

% Initializing graphicx
\usepackage{graphicx}
\graphicspath{{assets/}}

% Custom Commands
\newcommand{\img}[1] {
	\includegraphics[scale=0.33]{#1}
}

\title{Maman 11}
\author{Jonathan Ohayon}

\begin{document}
\maketitle
\setcounter{section}{-1}

\section{Declarations}
\subsection{Infinite Set}
\begin{center}
\emph{A set of elements E is said to be infinite if the elements of a proper subset S can be put into one-to-one correspondence with the elements of E.\footnote{Taken from \url{http://mathworld.wolfram.com/InfiniteSet.html}, just for the English version of this definition.}}
\end{center}

\section{Question 1}
\subsection{A}
Yes. Using the following function:
\begin{equation*}
f(x) \rightarrow \frac{2}{x/2}
\end{equation*}
Then:
\begin{equation*}
x_1, x_2 \in A \Rightarrow
x_1 \neq x_2 \Rightarrow
\frac{x_1}{2} \neq \frac{x_2}{2} \Rightarrow
\frac{2}{x_1/2} \neq \frac{2}{x_2/2}
\end{equation*}
$\bigotimes$
\clearpage

\subsection{B}
No. For example, using the following function:
\begin{equation*}
f(x) \rightarrow 8
\end{equation*}
Every $x \in A$ is matched to \emph{8}.\\
$\bigotimes$

\subsection{C}
Yes, using the function declared in answer A ($f(x) \rightarrow \frac{2}{x/2}$), $\frac{2}{1} \in B$ (which is 2) is matched to $2 \in A$.\\
$\bigotimes$

\subsection{D}
Yes. Let \emph{E} be the following set:
\begin{equation*}
E = B\backslash\{\frac{2}{1}\}
\end{equation*}
We'll show that $E \sim B$ with this one-to-one correspondence:
\begin{equation*}
f(x) \rightarrow \frac{2}{x/2+1}
\end{equation*}
That way, 2 is matched to $\frac{2}{2}$, 4 to $\frac{2}{3}$ and so on and so forth.\\
Since \emph{E} is a proper subset of \emph{B} ($E \subset B$) because it equals to \emph{B} without it's first element, and since \emph{E} has one-to-one correspondence with \emph{B} ($E \sim B$), \emph{B} is infinite.\\
$\bigotimes$

\section{Question 2}
\begin{center}
\begin{tabular}{ c c c }
\img{A} & \img{B} & \img{C}\\
A & B & C\\
\img{D} & \img{E}\\
D & E\\
\end{tabular}
\end{center}

\section{Question 3}
\subsection{A}
Yes. Let \emph{A} and \emph{B} be the following sets:
\begin{equation*}
A = \{1, 2, 3, 4\}, B = \{1, 2\}
\end{equation*}
Then:
\begin{equation*}
A \cup B = \{1, 2, 3, 4\}\Rightarrow
A \backslash B = \{3, 4\}\Rightarrow
A \cup B \neq A \backslash B
\end{equation*}
Therefore, \emph{B} has to be $\emptyset$.\\
$\bigotimes$

\subsection{B}
No. Let \emph{A} and \emph{B} be the following sets:
\begin{equation*}
A = \mathbb{N}, B = \{2, 4, 6, \ldots\}
\end{equation*}
Then $A \cup B \sim A \backslash B$, using the following function:
\begin{equation*}
f(x) \rightarrow 2x
\end{equation*}
Every $x \in A$ has one arrow sent from it to $2x \in B$, and every $y \in B$ has an arrow from $\frac{1}{2}y \in A$.\\
Therefore, \emph{B} doesn't have to be $\emptyset$.\\
$\bigotimes$

\subsection{C}
Yes. Let \emph{A}, \emph{B}, \emph{C} and \emph{D} be the following sets:
\begin{equation*}
\begin{split}
& A = \{1, 2, 3, 4, 5\}\\
& B = \{3, 4, 5\}\\
& C = A \cup B = \{1, 2, 3, 4, 5\}\\
& D = A \backslash B = \{1, 2\}
\end{split}
\end{equation*}
Then $D \nsim C$ because no there is no one-to-one correspondence between \emph{D} and \emph{C}, since the number of elements in \emph{D} is smaller than the number of elements in \emph{C}.\\
Therefore, \emph{B} has to be $\emptyset$.\\
$\bigotimes$
\clearpage

\section{Question 4}
\subsection{A}
No. Let \emph{A} and \emph{B} be the following sets:
\begin{equation*}
A = \{2, 3, 4, 5\}, B = \{5, 6, 7, 8\}
\end{equation*}
Then $B = A\backslash\{1\}$ and $A \sim B$ (with this function: $f(x) \rightarrow x + 3$), yet \emph{A} is finite (since it has 4 elements, and 4 isn't infinity).\\
Therefore, \emph{A} doesn't have to be infinite.\\
$\bigotimes$

\subsection{B}
Yes. \emph{B} is a proper subset of \emph{A} ($B \subset A$, since $A \neq B$), and according to the question's declarations, \emph{A} has one-to-one correspondence to \emph{B} ($A \sim B$).\\
Therefore, \emph{A} has to be infinite.\\
$\bigotimes$

\subsection{C}


\end{document}