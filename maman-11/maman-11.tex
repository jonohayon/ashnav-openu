\documentclass[12pt, oneside]{article}
\usepackage{geometry}
\geometry{letterpaper}
\usepackage{amssymb}
\usepackage{amsmath}
\usepackage{array}

% Initializing graphicx
\usepackage{graphicx}
\graphicspath{{assets/}}

% Custom Commands
\newcommand{\img}[1] {
	\includegraphics[scale=0.33]{#1}
}

\title{Maman 11}
\author{Jonathan Ohayon}

\begin{document}
\maketitle

\section{Question 1}
\subsection{A}
Yes. Using the following injective function:
\begin{equation*}
f(x) \rightarrow \frac{2}{x/2}
\end{equation*}
Then:
\begin{equation*}
x_1, x_2 \in A \Rightarrow
x_1 \neq x_2 \Rightarrow
\frac{x_1}{2} \neq \frac{x_2}{2} \Rightarrow
\frac{2}{x_1/2} \neq \frac{2}{x_2/2}
\end{equation*}
$\bigotimes$

\subsection{B}
No. For example, using the following function:
\begin{equation*}
f(x) \rightarrow 8
\end{equation*}
Every $x \in A$ is matched to \emph{8}.\\
$\bigotimes$
\clearpage

\section{Question 2}
\begin{center}
\begin{tabular}{ c c c }
\img{A} & \img{B} & \img{C}\\
A & B & C\\
\img{D} & \img{E}\\
D & E\\
\end{tabular}
\end{center}

\section{Question 3}
\subsection{A}
Yes. Let \emph{A} and \emph{B} be the following sets:
\begin{equation*}
A = \{1, 2, 3, 4\}, B = \{1, 2\}
\end{equation*}
Then:
\begin{equation*}
A \cup B = \{1, 2, 3, 4\}\Rightarrow
A \backslash B = \{3, 4\}\Rightarrow
A \cup B \neq A \backslash B
\end{equation*}\\
Therefore, \emph{B} has to be $\emptyset$.\\
$\bigotimes$

\subsection{B}
No. Let \emph{A} and \emph{B} be the following sets:
\begin{equation}
A = \mathbb{N}, B = \{2, 4, 6, \ldots\}
\end{equation}
Then $A \cup B \sim A \backslash B$, using the following injective function:
\begin{equation*}
f(x) = 2x
\end{equation*}
Every $x \in A$ has one arrow sent from it to $2x \in B$, and every $y \in B$ has an arrow from $\frac{1}{2}y \in A$.\\
Therefore, \emph{B} doesn't have to be $\emptyset$.\\
$\bigotimes$

\subsection{C}
Yes. Let \emph{A}, \emph{B}, \emph{C} and \emph{D} be the following sets:
\begin{equation*}
\begin{split}
& A = \{1, 2, 3, 4, 5\}\\
& B = \{3, 4, 5\}\\
& C = A \cup B = \{1, 2, 3, 4, 5\}\\
& D = A \backslash B = \{1, 2\}
\end{split}
\end{equation*}
Then $D \nsim C$ because no injective function could match \emph{D}'s members with \emph{C}'s members, since the number of members in \emph{D} is smaller than the number of members in \emph{C}.\\
Therefore, \emph{B} has to be $\emptyset$.\\
$\bigotimes$

\section{Question 4}
\subsection{A}


\end{document}