\documentclass[12pt, oneside]{article}
\usepackage{geometry}
\geometry{letterpaper}
\usepackage{amssymb}
\usepackage{amsmath}
\usepackage{hyperref}

% Initializing graphicx
\usepackage{graphicx}
\graphicspath{{assets/}}

% Custom Commands
\newcommand{\img}[1] {
	\includegraphics[scale=0.33]{#1}
}

\title{Maman 12}
\author{Jonathan Ohayon}
\date{November 27, 2016}

\begin{document}
\maketitle

\section{Question 1}
\setcounter{subsection}{-1}
\subsection{Declarations}
Let \emph{A} and \emph{B} be the following sets:
\begin{equation*}
A = \{1, \{2\}, \emptyset\}, B = \{1, 2\}
\end{equation*}

\subsection{A}
Answer:
\begin{equation*}
\begin{split}
& P(A) = \{\emptyset, \{1\}, \{\{2\}\}, \{\emptyset\}, \{1, \{2\}\}, \{\{2\}, \emptyset\}, \{1, \emptyset\}, A\}\\
& P(B) = \{\emptyset, \{1\}, \{2\}, B\}
\end{split}
\end{equation*}

\subsection{B}
Answer:
\begin{equation*}
\begin{split}
& P(A) \backslash P(B) = \{\{\{2\}\}, \{\emptyset\}, \{1, \{2\}\}, \{\{2\}, \emptyset\}, \{1, \emptyset\}, A\}\\
& P(B) \backslash ֿP(A) = \{\{2\}, B\}
\end{split}
\end{equation*}
\clearpage

\subsection{C}
Let $P(\emptyset)$ and $P(P(\emptyset))$ be the following sets:
\begin{equation*}
\begin{split}
& P(\emptyset) = \{\emptyset\}\\
& P(P(\emptyset)) = \{\emptyset, \{\emptyset\}\}
\end{split}
\end{equation*}
The number of elements in $P(\emptyset)$ is 1 ($2^0$, since there are no elements in $\emptyset$), and the number of elements in $P(P(\emptyset))$ is 2 ($2^1$, since the number of elements in $P(\emptyset)$ is 1).\\
Therefore, there cannot be a one to one correspondence between $P(\emptyset)$ and $P(P(\emptyset))$ ($P(\emptyset) \not\sim P(P(\emptyset))$).

\section{Question 2}
\subsection{A}
Let \emph{D} and \emph{E} be the following sets and \emph{x}:
\begin{equation*}
D = (A \cup B) \cap (C \backslash A), E = (B \backslash A) \cap C, x \in D
\end{equation*}
Let's assume that $D \neq E$, in order to prove this using contradiction.\\
If $x \in D$, then:
\begin{equation*}
\begin{split}
& x \in A \cup B\;\text{and}\;x \in C \backslash A \Rightarrow \\
& x \in A\;\text{or}\;B\;\text{and}\;x \in C\;\text{but}\;x \not\in A \Rightarrow \\
& x \in B, C\;\text{and}\; x \not\in A\Rightarrow \\
& x \in B \backslash A, \Rightarrow \\
& x \in (B \backslash A) \cap C
\end{split}
\end{equation*}
Therefore, $D \subseteq E$.\\
If $x \in E$:
\begin{equation*}
\begin{split}
& x \in B \backslash A\;\text{and}\;x \in C \Rightarrow \\
& x \in B, C\;\text{and}\;x \not\in A \Rightarrow \\
& x \in A \cup B\;\text{and}\;x \in C \Rightarrow \\
& x \in (A \cup B) \cap (C \backslash A)
\end{split}
\end{equation*}
Therefore, $E \subseteq D$.\\
According to the results, $E \subseteq D$ and $D \subseteq E$. Therefore there's a contradiction between our assumption and the question's constants, and therefore $D = E$.

\subsection{B}
Yes. Let's assume that $(A \backslash C) \cap B \neq \emptyset$, in order to prove this using contradiction. Then:
\begin{equation*}
\begin{split}
& A \backslash C \neq \emptyset \Rightarrow \\
& B \neq \emptyset \Rightarrow \\
& A \neq \emptyset \Rightarrow \\
& A \cup B \neq \emptyset \Rightarrow \\
& A \cup B \not\subseteq A \backslash B \Rightarrow \\
& (A \cup B) \backslash C \not\subseteq A \backslash B
\end{split}
\end{equation*}
Since we got a contradiction between the result and the question's constants, the assumption that $(A \backslash C) \cap B = \emptyset$ is correct.

\subsection{C}
Yes. Let $E \subseteq A$. Then:
\begin{equation*}
E \in P(A) \Rightarrow E \subseteq B \cap C \Rightarrow E \in P(B) \cap P(C) \Rightarrow P(A) = P(B) \cap P(C)
\end{equation*}

\subsection{D}
Yes. In order to show that $P(A \backslash B) = P(A) \backslash P(B)$, we'll show a two way substitution.
For every $E \in P(A \backslash B)$:
\begin{equation*}
E \in P(A \backslash B) \Rightarrow E \in P(A)\;\text{and}\;E \not\in P(B) \Rightarrow E \in P(A) \backslash P(B)
\end{equation*}
Therefore, $P(A \backslash B) \subseteq P(A) \backslash P(B)$.\\
\hfill\newline
For every $K \in P(A) \backslash P(B)$:
\begin{equation*}
K \in P(A) \backslash P(B) \Rightarrow K \in P(A)\;\text{and}\;E \not\in P(B) \Rightarrow K \in P(A \backslash B)
\end{equation*}
Therefore, $P(A) \backslash P(B) \subseteq P(A \backslash B)$.\\
\hfill\newline
In conclusion, because there's a two way substitution between $P(A) \backslash P(B)$ and $P(A \backslash B)$, $P(A \backslash B) = P(A) \backslash P(B)$.
\clearpage

\section{Question 3}
\subsection{A}
\setcounter{subsubsection}{-1}
\subsubsection{Declarations}
Let \emph{A} be a set with at least 2 elements that belong to it, and a binary operation * defined like so:
\begin{center}
For every $a, b \in A\;,\;a * b = b$.
\end{center}

\subsubsection{Closure}
In order to show that the * operation implements closure over \emph{A}, we need to show that it's results belong to the \emph{A} set. Therefore, we can use the sentence which we used to define the operation to show that:
\begin{center}
For every $a, b \in A\;,\;a * b = b$
\end{center}
Since the result of $a * b$ is \emph{b}, which belongs to A, the operation does implement closure.

\subsubsection{Associative property}
In order to show that the * operation has the associative property over \emph{A}, we need to show that the results of $a * (b * c)$ and $(a * b) * c$ are equal. We can show that using the following explanation:
\begin{equation*}
\begin{split}
& a, b, c \in A\\
& a * (b * c) = a * c = c\\
& (a * b) * c = b * c = c\\
\end{split}
\end{equation*}
Since both $a * (b * c)$ and $(a * b) * c$ equal \emph{c}, the operation has the associative property.
\clearpage

\subsubsection{Commutativity}
In order to show that the * operation implements commutativity over \emph{A}, we need to show that the results of $a * b$ and $b * a$ are equal for every $a, b \in A$. We can show that doesn't happen over \emph{A} using the following counter example:
\begin{equation*}
\begin{split}
& A = \mathbb{N}\\
& 1 * 2 = 2\\
& 2 * 1 = 1\\
& 2 \neq 1
\end{split}
\end{equation*}
Since $2 \neq 1$, the * operation does not implement commutativity.

\subsubsection{Neutral element}
In order to show that there's a neutral element in \emph{A} under the * operation, we need to show an element of \emph{A} that, when being used in the * operation, returns the other element of \emph{A} (aka $x * e = e * x = x$ for every $x \in A$). We can show that \emph{A} doesn't have a neutral element inside of it using the following counter example:
\begin{center}
$A = \{a, b, c\}$\\
\hfill\break
\begin{tabular}{c|c|c|c}
* & a & b & c\\
\hline
a & a & b & c\\
\hline
b & a & b & c\\
\hline
c & a & b & c
\end{tabular}
\end{center}
Since the * operation returns the number to the right of the operator by definition, there isn't any element in the given \emph{A} set that can be used on every other element and return the latter.
\clearpage

\subsection{B}

\end{document}